% ------------------------------------------------------------------------
% AMS-LaTeX Paper ********************************************************
% ------------------------------------------------------------------------
% Submitted:      Dec 15 2003
% Final Version:  
% Accepted:       
% ------------------------------------------------------------------------
% This is a journal top-matter template file for use with AMS-LaTeX.
%%%%%%%%%%%%%%%%%%%%%%%%%%%%%%%%%%%%%%%%%%%%%%%%%%%%%%%%%%%%%%%%%%%%%%%%%%

% TODO(mike): locating a closure at a name: how to translate a closure into a process that gets stuck into the cell



%\documentclass{tran-l}
%\documentclass[twocolumn]{amsart}
%\documentclass[]{amsart}
%\documentclass[]{sig-alternate}
%\documentclass[fleqn]{acm_proc_article-sp}
\documentclass[]{acm_proc_article-sp}
%\documentclass[]{llncs}


%\documentclass[]{prentcsmacro}

%\usepackage[active]{srcltx} % SRC Specials for DVI Searching
\usepackage{url}
% \usepackage[pdf]{pstricks}
% \usepackage{pstricks-add, pst-grad, pst-plot}
% \usepackage[tiling]{pst-fill}
% \psset{linewidth=0.3pt,dimen=middle}
% \psset{xunit=.70cm,yunit=0.70cm}
% \psset{angleA=-90,angleB=90,ArrowInside=->,arrowscale=2}


% From Allen's stable.
\usepackage{bigpage}
\usepackage{bcprules}
%\usepackage{code}
\usepackage{mathpartir}
\usepackage{listings}
\usepackage{mathtools}
%\usepackage[fleqn]{amsmath}
\usepackage{amsfonts}
\usepackage{cmll}
\usepackage{latexsym}
\usepackage{amssymb}
\usepackage{caption}
%\usepackage{multicol}


% Math
\newcommand{\maps}{\colon}
\newcommand{\NN}{\mathbb{N}}
% Double brackets
\newcommand{\ldb}{[\![}
\newcommand{\rdb}{]\!]}
\newcommand{\ldrb}{(\!(}
\newcommand{\rdrb}{)\!)}
\newcommand{\lliftb}{\langle\!|}
\newcommand{\rliftb}{|\!\rangle}
% \newcommand{\lpquote}{\langle}
% \newcommand{\rpquote}{\rangle}
% \newcommand{\lpquote}{\lceil}
% \newcommand{\rpquote}{\rceil}
\newcommand{\lpquote}{\ulcorner}
\newcommand{\rpquote}{\urcorner}
\newcommand{\newkw}{\nu}

% SYNTAX
\newcommand{\id}[1]{\texttt{#1}}
\newcommand{\none}{\emptyset}
\newcommand{\eps}{\epsilon}
\newcommand{\set}[1]{\{#1\}}
\newcommand{\rep}[2]{\id{\{$#1$,$#2$\}}}
\newcommand{\elt}[2]{\id{$#1$[$#2$]}}
\newcommand{\infinity}{$\infty$}

\newcommand{\pzero}{\mathbin{0}}
\newcommand{\seq}{\mathbin{\id{,}}}
\newcommand{\all}{\mathbin{\id{\&}}}
\newcommand{\choice}{\mathbin{\id{|}}}
\newcommand{\altern}{\mathbin{\id{+}}}
\newcommand{\juxtap}{\mathbin{\id{|}}}
%\newcommand{\concat}{\mathbin{.}}
\newcommand{\concat}{\Rightarrow}
\newcommand{\punify}{\mathbin{\id{:=:}}}
\newcommand{\fuse}{\mathbin{\id{=}}}
\newcommand{\scong}{\mathbin{\equiv}}
\newcommand{\nameeq}{\mathbin{\equiv_N}}
\newcommand{\alphaeq}{\mathbin{\equiv_{\alpha}}}
\newcommand{\names}[1]{\mathbin{\mathcal{N}(#1)}}
\newcommand{\freenames}[1]{\mathbin{\mathcal{FN}(#1)}}
\newcommand{\boundnames}[1]{\mathbin{\mathcal{BN}(#1)}}
%\newcommand{\lift}[2]{\texttt{lift} \; #1 \concat #2}
\newcommand{\binpar}[2]{#1 \juxtap #2}
\newcommand{\outputp}[2]{#1 ! ( * #2 )}
\newcommand{\prefix}[3]{#1 ? ( #2 ) \concat #3}
\newcommand{\lift}[2]{#1 ! ( #2 )}
%\newcommand{\quotep}[1]{\lpquote #1 \rpquote}
\newcommand{\quotep}[1]{@#1}
\newcommand{\dropn}[1]{*#1}

\newcommand{\newp}[2]{\id{(}\newkw \; #1 \id{)} #2}
\newcommand{\bangp}[1]{\int #1}
\newcommand{\xbangp}[2]{\int_{#2} #1}
\newcommand{\bangxp}[2]{\int^{#2} #1}

\newcommand{\substp}[2]{\id{\{} \quotep{#1} / \quotep{#2} \id{\}}}
\newcommand{\substn}[2]{\id{\{} #1 / #2 \id{\}}}

\newcommand{\psubstp}[2]{\widehat{\substp{#1}{#2}}}
\newcommand{\psubstn}[2]{\widehat{\substn{#1}{#2}}}

\newcommand{\applyp}[2]{#1 \langle #2 \rangle}
\newcommand{\absp}[2]{\id{(} #1 \id{)} #2}

\newcommand{\transitions}[3]{\mathbin{#1 \stackrel{#2}{\longrightarrow} #3}}
\newcommand{\meaningof}[1]{\ldb #1 \rdb}
\newcommand{\pmeaningof}[1]{\ldb #1 \rdb}
\newcommand{\nmeaningof}[1]{\ldrb #1 \rdrb}

\newcommand{\Proc}{\mathbin{Proc}}
\newcommand{\QProc}{\quotep{\mathbin{Proc}}}

\newcommand{\entailm}{\mathbin{\vdash_{\mathfrak m}}} %matching
\newcommand{\entailp}{\mathbin{\vdash_{\mathfrak p}}} %behavioral
\newcommand{\entailv}{\mathbin{\vdash_{\mathfrak v}}} %validation
\newcommand{\congd}{\mathbin{\equiv_{\mathfrak d}}}
\newcommand{\congs}{\mathbin{\equiv_{\mathfrak s}}}
\newcommand{\congp}{\mathbin{\equiv_{\mathfrak p}}}
%\newcommand{\defneqls}{:\!=}
\newcommand{\defneqls}{\coloneqq}
%\newcommand{\logequiv}{\mathbin{\leftrightarrow}}

\newcommand{\barb}[2]{\mathbin{#1 \downarrow_{#2}}}
\newcommand{\dbarb}[2]{\mathbin{#1 \Downarrow_{#2}}}

% From pi-duce paper
\newcommand{\red}{\rightarrow}
\newcommand{\wred}{\Rightarrow}
\newcommand{\redhat}{\hat{\longrightarrow}}
\newcommand{\lred}[1]{\stackrel{#1}{\longrightarrow}} %transitions
\newcommand{\wlred}[1]{\stackrel{#1}{\Longrightarrow}}

\newcommand{\opm}[2]{\overline{#1} [ #2 ]} % monadic
\newcommand{\ipm}[2]{{#1} ( #2 )} 
\newcommand{\ipmv}[2]{{#1} ( #2 )} % monadic
\newcommand{\parop}{\;|\;}    % parallel operator
\newcommand{\patmatch}[3]{#2 \in #3 \Rightarrow #1}
\newcommand{\sdot}{\, . \,}    % Space around '.'
\newcommand{\bang}{!\,}
%\newcommand{\fuse}[1]{\langle #1 \rangle}    
\newcommand{\fusion}[2]{#1 = #2} % fusion prefix/action
\newcommand{\rec}[2]{\mbox{\textsf{rec}} \, #1. \, #2}
\newcommand{\match}[2]{\mbox{\textsf{match}} \; #1 \; \mbox{\textsf{with}} \; #2}
\newcommand{\sep}{:}
\newcommand{\val}[2]{\mbox{\textsf{val}} \; #1 \; \mbox{\textsf{as}} \; #2}

\newcommand{\rel}[1]{\;{\mathcal #1}\;} %relation
\newcommand{\bisim}{\stackrel{.}{\sim}_b} %bisimilar
\newcommand{\wb}{\approx_b} %weak bisimilar
\newcommand{\bbisim}{\stackrel{\centerdot}{\sim}} %barbed bisimilar
\newcommand{\wbbisim}{\stackrel{\centerdot}{\approx}} %weak barbed bisimilar
\newcommand{\wbbisimsem}{\approx} %weak barbed bisimilar
\newcommand{\bxless}{\lesssim}  %expansion less (amssymb required)
\newcommand{\bxgtr}{\gtrsim}  %expansion greater (amssymb required)
\newcommand{\beq}{\sim}    %barbed congruent
\newcommand{\fwbeq}{\stackrel{\circ}{\approx}}  %weak barbed congruent
\newcommand{\wbeq}{\approx}  %weak barbed congruent
\newcommand{\sheq}{\simeq}  %symbolic hypereq
\newcommand{\wbc}{\approx_{cb}}

% End piduce contribution

% rho logic

\newcommand{\ptrue}{\mathbin{true}}
\newcommand{\psatisfies}[2]{#1 \models #2}
\newcommand{\pdropf}[1]{\rpquote #1 \lpquote}
\newcommand{\pquotep}[1]{\lpquote #1 \rpquote}
\newcommand{\plift}[2]{#1 ! ( #2 )}
\newcommand{\pprefix}[3]{\langle #1 ? #2 \rangle #3}
\newcommand{\pgfp}[2]{\textsf{rec} \; #1 \mathbin{.} #2}
\newcommand{\pquant}[3]{\forall #1 \mathbin{:} #2 \mathbin{.} #3}
\newcommand{\pquantuntyped}[2]{\forall #1 \mathbin{.} #2}
\newcommand{\riff}{\Leftrightarrow}

\newcommand{\PFormula}{\mathbin{PForm}}
\newcommand{\QFormula}{\mathbin{QForm}}
\newcommand{\PropVar}{\mathbin{\mathcal{V}}}

\newcommand{\typedby}{\mathbin{\:\colon}}
\newcommand{\mixedgroup}[1]{\id{mixed($#1$)}}
\newcommand{\cast}[2]{\id{CAST AS} \; #1 \; (#2)}
\newcommand{\bslsh}{\mathbin{\id{\\}}}
\newcommand{\bslshslsh}{\mathbin{\id{\\\\}}}
\newcommand{\fslsh}{\mathbin{\id{/}}}
\newcommand{\fslshslsh}{\mathbin{\id{//}}}
\newcommand{\bb}[1]{\mbox{#1}}
\newcommand{\bc}{\mathbin{\mathbf{::=}}}
\newcommand{\bm}{\mathbin{\mathbf\mid}}
\newcommand{\be}{\mathbin{=}}
\newcommand{\bd}{\mathbin{\buildrel {\rm \scriptscriptstyle def} \over \be}}
\newcommand{\ctcategory}[1]{\mbox{\bf #1}}

%GRAMMAR
\newlength{\ltext}
\newlength{\lmath}
\newlength{\cmath}
\newlength{\rmath}
\newlength{\rtext}

\settowidth{\ltext}{complex type name}
\settowidth{\lmath}{$xxx$}
\settowidth{\cmath}{$::=$}
\settowidth{\rmath}{\id{attributeGroup}}
\settowidth{\rtext}{repetition of $g$ between $m$ and $n$ times}

\newenvironment{grammar}{
  \[
  \begin{array}{l@{\quad}rcl@{\quad}l}
  \hspace{\ltext} & \hspace{\lmath} & \hspace{\cmath} & \hspace{\rmath} & \hspace{\rtext} \\
}{
  \end{array}\]
}

% Over-full v-boxes on even pages are due to the \v{c} in author's name
\vfuzz2pt % Don't report over-full v-boxes if over-edge is small

% THEOREM Environments ---------------------------------------------------
 \newtheorem{thm}{Theorem}[subsection]
 \newtheorem{cor}[thm]{Corollary}
 \newtheorem{lem}[thm]{Lemma}
 \newtheorem{prop}[thm]{Proposition}
% \theoremstyle{definition}
 \newtheorem{defn}[thm]{Definition}
% \theoremstyle{remark}
 \newtheorem{rem}[thm]{Remark}
 \newtheorem{example}[thm]{Example}
 \numberwithin{equation}{subsection}
% MATH -------------------------------------------------------------------
 \DeclareMathOperator{\RE}{Re}
 \DeclareMathOperator{\IM}{Im}
 \DeclareMathOperator{\ess}{ess}
 \newcommand{\veps}{\varepsilon}
 \newcommand{\To}{\longrightarrow}
 \newcommand{\h}{\mathcal{H}}
 \newcommand{\s}{\mathcal{S}}
 \newcommand{\A}{\mathcal{A}}
 \newcommand{\J}{\mathcal{J}}
 \newcommand{\M}{\mathcal{M}}
 \newcommand{\W}{\mathcal{W}}
 \newcommand{\X}{\mathcal{X}}
 \newcommand{\BOP}{\mathbf{B}}
 \newcommand{\BH}{\mathbf{B}(\mathcal{H})}
 \newcommand{\KH}{\mathcal{K}(\mathcal{H})}
 \newcommand{\Real}{\mathbb{R}}
 \newcommand{\Complex}{\mathbb{C}}
 \newcommand{\Field}{\mathbb{F}}
 \newcommand{\RPlus}{\Real^{+}}
 \newcommand{\Polar}{\mathcal{P}_{\s}}
 \newcommand{\Poly}{\mathcal{P}(E)}
 \newcommand{\EssD}{\mathcal{D}}
 \newcommand{\Lom}{\mathcal{L}}
 \newcommand{\States}{\mathcal{T}}
 \newcommand{\abs}[1]{\left\vert#1\right\vert}
% \newcommand{\set}[1]{\left\{#1\right\}}
%\newcommand{\seq}[1]{\left<#1\right>}
 \newcommand{\norm}[1]{\left\Vert#1\right\Vert}
 \newcommand{\essnorm}[1]{\norm{#1}_{\ess}}

%%% NAMES
\newcommand{\Names}{{\mathcal N}}
\newcommand{\Channels}{{\sf X}}
\newcommand{\Variables}{{\mathcal V}}
\newcommand{\Enames}{{\mathcal E}}
\newcommand{\Nonterminals}{{\mathcal S}}
\newcommand{\Pnames}{{\mathcal P}}
\newcommand{\Dnames}{{\mathcal D}}
\newcommand{\Types}{{\mathcal T}}

\newcommand{\fcalc}{fusion calculus}
\newcommand{\xcalc}{${\mathfrak x}$-calculus}
\newcommand{\lcalc}{$\lambda$-calculus}
\newcommand{\pic}{$\pi$-calculus}
\newcommand{\rhoc}{${\textsc{rho}}$-calculus}
\newcommand{\hcalc}{highwire calculus}
\newcommand{\dcalc}{data calculus}
%XML should be all caps, not small caps. --cb
%\newcommand{\xml}{\textsc{xml}}
\newcommand{\xml}{XML} 

\newcommand{\papertitle}{Formally introducing Casper}
% use static date to preserve date of actual publication
 \newcommand{\paperversion}{Draft Version 0.1 - Jul 30, 2015}

\newenvironment{toc}
{
\begin{list}{}{
   \setlength{\leftmargin}{0.4in}
   \setlength{\rightmargin}{0.6in}
   \setlength{\parskip}{0pt}
 } \item }
{\end{list}}

\newenvironment{narrow}
{
\begin{list}{}{
   \setlength{\leftmargin}{0.4in}
   \setlength{\rightmargin}{0.6in}
 } \item }
{\end{list}}

%%% ----------------------------------------------------------------------

%\title{\huge{\papertitle}}
\title{\papertitle}

%\numberofauthors{3}
\author{
Vlad Zamfir\\
  \affaddr{Ethereum}\\
  \email{\fontsize{8}{8}\selectfont vldzmfr@gmail.com}
\and
L.G. Meredith\\
  \affaddr{CSO, Synereo}\\
  \email{\fontsize{8}{8}\selectfont greg@synereo.com}
}

%\address{Systems Biology, Harvard Medical School, Boston, Massachussetts, USA}

%\email{lg_meredith@hms.harvard.edu}

%\thanks{This work was completed during a visiting appointment at the Department of Systems Biology, Harvard Medical School.}

%\subjclass{Primary 47A15; Secondary 46A32, 47D20}

%\date{April 6, 2002.}

%\dedicatory{}

%\commby{Daniel J. Rudolph}

%%% ----------------------------------------------------------------------

\begin{document}
%\lstset{language=erlang}
\lstset{language=}

%These margin values appear to be relative to the bigpage package settings. --cb
\setlength{\topmargin}{0in}
\setlength{\textheight}{8.5in}
\setlength{\parskip}{6pt}

\keywords{ {\pic}, proof-of-stake, blockchain, types, Curry-Howard }

\begin{abstract}
\normalsize{ 

  We present Casper, a proof-of-stake protocol, and it's formal specification in {\pic}.

}

\end{abstract}

% \noindent
% {\large \textbf{Submission to arXiv}}\\
% \rule{6.25in}{0.75pt}\\\\\\

%%% ----------------------------------------------------------------------
\maketitle
%%% ----------------------------------------------------------------------

% \begin{center}
% \paperversion\\
% \end{center}

% \begin{toc}
% \tableofcontents
% \end{toc}

% \newpage
% ------------------------------------------------------------------------

\section{Introduction and Motivation}

\subsubsection{Related work}

\subsubsection{Organization of the rest of the paper}

\section{Casper, informally}

\section{The calculus}

One notable feature of the {\pic} is its ability to succinctly and
faithfully model a number of phenomena of concurrent and distributed
computing. Competition for resources amongst autonomously executing
processes is a case in point. The expression
\begin{equation*}
  x?( y ) \Rightarrow P \juxtap x!( u ) \juxtap x?( v ) \Rightarrow Q
\end{equation*}
is made by composing three processes, two of which, $x?( y )
\Rightarrow P$ and $x?( v ) \Rightarrow Q$ are seeking input from
channel $x$ before they launch their respective continuations, $P$
and/or $Q$; while the third, $x!( u )$, is supplying output on that
same said channel. Only one of the input-guarded processes will win,
receiving $u$ and binding it to the input variable, $y$, or
respectively, $v$ in the body of the corresponding continuation --
while the loser remains in the input-guarded state awaiting input
along channel $x$. The calculus is equinanimous, treating both
outcomes as equally likely, and in this regard is unlike its
sequential counterpart, the $\lambda$-calculus, in that it is not
\emph{confluent}. There is no guarantee that the different branches of
computation must eventually converge. Note that just adding a
$\mathsf{new}$-scope around the expression
\begin{equation*}
  (\mathsf{new}\; x)( x?( y ) \Rightarrow P \juxtap x!( u ) \juxtap x?( v ) \Rightarrow Q )
\end{equation*}
ensures that the competition is for a local resource, hidden from any
external observer.

\subsection{Our running process calculus}

\subsubsection{Syntax}
\label{syntax}
\begin{grammar}
{P} \bc \pzero & \mbox{stopped process} \\
       \;\;\; \bm \; {x}{?}{( y_1, \ldots, y_n )} \Rightarrow {P} & \mbox{input} \\
       \;\;\; \bm \; {x}{!}{( y_1, \ldots, y_n )} & \mbox{output} \\
%       \;\;\; \bm \; {M}{+}{N} & \mbox{choice} \\
%{ P, Q } \bc M & \mbox{include IO processes} \\                                
       \;\;\; \bm \; (\mathsf{new}\; x){P} & \mbox{new channel} \\
       \;\;\; \bm \; {P} \juxtap {Q} & \mbox{parallel} \\                                
\end{grammar}

Due to space limitations we do not treat replication, $!P$.

\subsubsection{Free and bound names}

\begin{equation*}
  \begin{aligned}
    & \freenames{\pzero} \defneqls \emptyset \\
    & \freenames{{x}{?}{( y_1, \ldots, y_n )} \Rightarrow {P}} \defneqls \\
    & \;\;\;\;\;\{ x \} \cup (\freenames{P} \setminus \{ y_1, \ldots y_n \}) \\
    & \freenames{{x}{!}{( y_1, \ldots, y_n )}} \defneqls \{ x, y_1, \ldots, y_n \} \\
    & \freenames{(\mathsf{new}\; x){P}} \defneqls \freenames{P} \setminus \{x\} \\    
    & \freenames{{P} \juxtap {Q}} \defneqls \freenames{P} \cup \freenames{Q} \\
  \end{aligned}
\end{equation*}

An occurrence of $x$ in a process $P$ is \textit{bound} if it is not
free. The set of names occurring in a process (bound or free) is
denoted by $\names{P}$.

\subsubsection{Structural congruence}
\label{congruence}

The {\em structural congruence} of processes, noted $\scong$, is the
least congruence containing $\alpha$-equivalence, $\alphaeq$, making
$( P, |, 0 )$ into commutative monoids and satisfying

\begin{equation*}  
  (\mathsf{new}\; x)(\mathsf{new}\; x){P} \scong (\mathsf{new}\; x)P
\end{equation*}
\begin{equation*}  
  (\mathsf{new}\; x)(\mathsf{new}\; y){P} \scong (\mathsf{new}\; y)(\mathsf{new}\; x)P
\end{equation*}
\begin{equation*}  
  ((\mathsf{new}\; x){P}) \juxtap {Q} \scong (\mathsf{new}\; x)({P} \juxtap {Q})
\end{equation*}

\subsubsection{Operational Semantics}\label{section:opsem}
 
\infrule[Comm]
{ |\vec{y}| = |\vec{z}| }
%{P_1 + {{ x_{0}{?}{(}{\vec{y}}{)} \concat {P}}\juxtap {x_{1}}{!}{(}{\vec{z}}{)} + P_2}
{{{ x{?}{(}{\vec{y}}{)} \concat {P}}\juxtap {x}{!}{(}{\vec{z}}{)}}
\red {{P}{\{}\vec{z}{/}{\vec{y}}{\}}}}

In addition, we have the following context rules:

\infrule[Par]{{P} \red {P}'}{{{P} \juxtap {Q}} \red {{P}' \juxtap {Q}}}

\infrule[New]{{P} \red {P}'}{{(\mathsf{new}\; x){P}} \red {(\mathsf{new}\; x){P}'}}

\infrule[Equiv]{{{P} \scong {P}'} \andalso {{P}' \red {Q}'} \andalso {{Q}' \scong {Q}}}{{P} \red {Q}}

\subsubsection{Bisimulation}

\begin{defn}
An \emph{observation relation}, $\downarrow$ is the smallest relation satisfying the rules
below.

\infrule[Out-barb]{ }
      {{x}!(\vec{y}) \downarrow x}
\infrule[Par-barb]{\mbox{$P\downarrow x$ or $Q\downarrow x$}}
      {{P} \juxtap {Q} \downarrow x}

% We write $P \Downarrow x$ if there is $Q$ such that 
% $P \wred Q$ and $Q \downarrow x$.
\end{defn}

Notice that $\prefix{x}{y}{P}$ has no barb.  Indeed, in {\pic} as well
as other asynchronous calculi, an observer has no direct means to
detect if a sent message has been received or not.

\begin{defn}
%\label{def.bbisim}
An \emph{barbed bisimulation}, is a symmetric binary relation 
${\mathcal S}$ between agents such that $P\rel{S}Q$ implies:
\begin{enumerate}
\item If $P \red P'$ then $Q \red Q'$ and $P'\rel{S} Q'$.
\item If $P\downarrow x$, then $Q\downarrow x$.
\end{enumerate}
$P$ is barbed bisimilar to $Q$, written
$P \wbbisim Q$, if $P \rel{S} Q$ for some barbed bisimulation ${\mathcal S}$.
\end{defn}

\section{Formally introducing Casper}

\paragraph{Acknowledgments}
We would like to acknowledge Anthony D'Onofrio for putting us in touch.

% ------------------------------------------------------------------------
%GATHER{Xbib.bib}   % For Gather Purpose Only
%GATHER{Paper.bbl}  % For Gather Purpose Only
\bibliographystyle{amsplain}
\bibliography{pispec}

% ------------------------------------------------------------------------

\section{Appendix: because every technical paper needs an appendix}



% ------------------------------------------------------------------------

\end{document}
% ------------------------------------------------------------------------
